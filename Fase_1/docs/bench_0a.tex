\documentclass{article}
\usepackage[utf8]{inputenc}
\usepackage[T1]{fontenc}
\usepackage[utf8]{inputenc}
\usepackage{lmodern}
\usepackage[a4paper, margin=1in]{geometry}
\usepackage{siunitx}
\usepackage{float}

\large
\title{Phase I}
\begin{document}
\begin{titlepage}
	\begin{center}
    \line(1,0){300}\\
    [0.65cm]
	\huge{\bfseries Strong formulation}\\
	\line(1,0){300}\\
	\textsc{\Large Geomech network: Benchmark 0a}\\
	\textsc{\LARGE \today}\\
	[5.5cm]     
	\end{center}
	\begin{flushright}
		\textsc{\Large LabMeC staff: TQ, OD, PC, PD}\\
		[0.5cm]
	\end{flushright}
\end{titlepage}

\section{Mass conservation}

We consider for both porous media and fractures the following assumptions:

\begin{itemize}
\item Steady state
\item Darcy flow 2D on porous media.
\item Darcy flow 1D on fractures.
\item Water as incompressible fluid.
\item No gravity.
\item Incompressible solid/rock.
\item Among others ...
\end{itemize}


The mass balance equation for 2D porous media is:

\begin{equation}
div\left(\rho\mathbf{v}\right)=0
\end{equation}

The Darcy's velocity for the porous media is:

\begin{equation}
\mathbf{v}=-\frac{1}{\mu}\mathbf{K}\nabla\left(p\right)
\end{equation}

The tensor $\mathbf{K}$ represents the absolute permeability $\si{\square\metre}$.
\begin{equation}
\mathbf{K}=\left[\begin{array}{cc}
k_{h} & 0\\
0 & k_{v}
\end{array}\right]
\end{equation}

The scalars $p$, $\rho$ and $\mu$ are fluid pressure $\si{\pascal}$, fluid density $\si{\kilo\gram\per\cubic\metre}$ and fluid viscosity $\si{\pascal\second}$ respectively.

The mass balance equation for 1D fracture is:

\begin{equation}
div\left(\rho v_{f}\right)+q_{f}=0
\end{equation}

The Darcy's velocity for fracture media is:

\begin{equation}
v_{x}=-\frac{1}{\mu}k_{h}\left(p\right)
\end{equation}

The scalar $k_{f}$ and $q_{f}$ represents the absolute permeability for the fracture in $\si{\square\metre}$ and cross section source term $\si{\kilo\gram\per\cubic\metre\second}$.

\section{Expected output}

\subsection*{Equivalent or average horizontal permeability}


Considering a layered reservoir, the cross section horizontal flow
rate form each layer can be calculated as:

Layer 1:

\begin{equation}
q_{x}=v_{x}A_{1}=-\frac{A_{1}}{\mu}k_{h}\frac{\left(p_{out}-p_{in}\right)}{\Delta x}
\end{equation}

Layer 2:

\begin{equation}
q_{f}=v_{f}A_{2}=-\frac{A_{2}}{\mu}k_{f}\frac{\left(p_{out}-p_{in}\right)}{\Delta x}
\end{equation}

The total flow rate is 

\begin{equation}
q_{tx}=v_{tx}A=-\frac{A}{\mu}k_{avgh}\frac{\left(p_{out}-p_{in}\right)}{\Delta x}
\end{equation}

or

\begin{equation}
q_{tx}=q_{x}+q_{f}=-\frac{A}{\mu}k_{avgh}\frac{\left(p_{out}-p_{in}\right)}{\Delta x}
\end{equation}

Leading to 

\begin{equation}
k_{avgh}=\frac{2 k_{h}A_{1}+k_{f}A_{2}}{2 A_{1}+A_{2}}
\end{equation}

\subsection*{Equivalent or average horizontal permeability}

The vertical average permeability $k_v$ remains unaltered.

\subsection*{Equivalent or average porosity}

The average porosity $\phi$ is updated as:

\begin{equation}
\phi_{avgh}=\frac{V_{p}+V_{f}}{V_{t}+V_{f}}
\end{equation}

\section*{Appendix neopz code:  Benchmark 0a}

\section{Expected output with provided data}

\begin{table}[H]
\begin{centering}
\begin{tabular}{|c|c|c|}
\hline 
Quantity & Given units & si units\tabularnewline
\hline 
\hline 
$k_{h}$ & 343.29 $mD$ & 3.38801e-13 $\si{\square\metre}$ \tabularnewline
\hline 
$k_{v}$ & 0.026 $mD$ & 2.56600e-17 $\si{\square\metre}$ \tabularnewline
\hline 
$k_{f}$ & 475000 $mD$ & 4.68789e-10 $\si{\square\metre}$ \tabularnewline
\hline 
$\phi$ & $0.0758$ & $0.0758$\tabularnewline
\hline 
$\Delta x$ & 200 $\si{\metre}$  & 200 $\si{\metre}$ \tabularnewline
\hline 
$\Delta y$ & 5 $\si{\metre}$  & 5 $\si{\metre}$ \tabularnewline
\hline 
$\Delta y_{f}$ & 6.5e-5 $\si{\metre}$  & 6.5e-5 $\si{\metre}$ \tabularnewline
\hline 
\end{tabular}
\par\end{centering}
\caption{Benchmark 0a data.}

\end{table}
\begin{table}[H]
\begin{centering}
\begin{tabular}{|c|c|c|}
\hline 
Quantity & Given units & si units\tabularnewline
\hline 
\hline 
$k_{avgh}$ & 346.375 $mD$ & 3.41845e-13 $\si{\square\metre}$ \tabularnewline
\hline 
$k_{avgv}$ & 0.026 $mD$ & 2.56600e-17 $\si{\square\metre}$ \tabularnewline
\hline 
$\phi_{avg}$ & $0.075806007$ & $0.075806007$\tabularnewline
\hline 
\end{tabular}
\par\end{centering}
\caption{Benchmark 0a results.}
\end{table}

\section{Discussion}

The expected output is completely determined by geometrical arguments.



\end{document}











